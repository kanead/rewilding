\documentclass[a4paper,12pt]{article}
\usepackage{jae}
\usepackage{natbib}
\title{Micromanaging Conservation -  \\ 
how microcosm experiments can inform rewilding campaigns}
\running{Rewilding with Microcosms}

\author{Adam Kane$^{1*}$, \and Deirdre McClean$^{2,3}$}

\affiliations{
\item University College Cork, Cooperage Building, School of Biological Earth and Environmental Sciences, Cork, Ireland.
\item Trinity College Dublin, Department of Zoology, School of Natural Sciences, Dublin 2, Ireland.
\item Trinity Centre for Biodiversity Research, Trinity College Dublin, Dublin 2, Ireland.
}

\nwords{xxxx}
\ntables{xxxx}
\nfig{xxxx}
\nref{xxxx}

\corr{\url{adam.kane@ucc.ie}}


\begin{document}


\maketitle


\begin{abstract}
Rewilding is a recently heralded approach to conservation that focuses on ecosystem restoration through species reintroductions. Yet, it suffers from disparate definitions and a lack of robust ecological theory. As rewilding typically focuses on large vertebrates, experiments investigating its potential at such scales are unfeasible. Top down effects from these reintroductions also tend to vary across systems, impacting our ability to generalise. 

Here, we highlight a hitherto untapped experimental approach in conservation science, in the form of microcosm experiments. These methods have the benefits of short generation times, ease of manipulation and a burgeoning research community. Indeed, they are widely employed in many other facets of ecology such as community ecology and climate change research.

Our review sets out four commonly used definitions of rewilding and describes specific microcosm experiments that could be used to gain a better understanding of each in terms of their ecological and evolutionary impact. The results from this microcosm perspective will put us in a better position to forecast many of the biological consequences of a rewilding campaign at full-scale, ultimately allowing us to evaluate the merit of this controversial strategy. 

 % \noindent \begin{enumerate}
 % \item This
 % \item Is
 % \item About
 % \item \emph{Synthesis and applications.} Rewilding 
 % \end{enumerate}
\end{abstract}

\noindent \textbf{Keywords:} rewilding, microcosms, conservation



\newpage


\section*{Introduction}
Introduction: 
Rewilding definition 
Maybe a lil Nate
Model systems 
Microcosms
Moreover, the credibility gap of their general applicability has eroded in light of impressive results with ecological and evolutionary relevance.

Headings/ Body:
Classical rewilding : 
What it is/ aims. Predators \\
Experiment suggestions: 

Pleistocene rewilding: Evo and eco potential; functional a la Britt Koskella? temporal experiments - could be interesting cos they actually show that things are too far gone to “readapt” to previous phages for example \\
Experiment suggestions: 

Passive rewilding: Have some human disturbance/ structure over a long period of time then remove it. \\
Experiment suggestions: 

Translocation rewilding : Quite similar to Pleistocene - I have in mind two expts, knock out some species/ function in expt 1 and after a time translocate it back in from expt 2. I suppose the nature of having two ongoing expts distinguishes this. \\
Experiment suggestions: 

\bigskip
Rewilding – the reintroduction of once native flora and fauna to an area which is then let to recover naturally – is a controversial idea \citep{monbiot2013feral}.  
It contrasts with the ‘protect what we’ve got’ tactic common to conservancy measures \citep{monbiot2013feral}. 
Leaving aside political barriers to restoring populations and habitats of extinct animals, the scientific obstacles are considerable. 
For instance, significant perturbations arise when an extinction or reintroduction event occurs, especially when it happens to be the loss of an apex predator \citep{mittelbach1995perturbation}, which are often the target of rewilding projects. 
Indeed, the reintroduction of grey wolves in the USA had a radical effect on biodiversity as trophic webs re-emerged and repaired which was seen with the increase in the beaver population as wolves preyed on the herbivores that damaged trees essential to the beavers’ survival \citep{hebblewhite2005human}.  
Many other species resurged once the wolves dampened the pressure from their coyote competitors. 
Investigations into the effects of such events can be carried out at the scale of the ecosystem, as was the case with the Yellowstone wolves, but often take years to complete \citep{mittelbach1995perturbation}.  
Moreover, taking into account every possible confound is impossible at this scale. 
These problems diminish at the level of a microcosm. 
So, in this study we will use microbial microcosms and community ecological theory to explore the dynamics of a rewilded system. 
This will involve the creation of a complex ecosystem at small scale, using microorganisms, to infer the ecology of large scale systems. 
The manipulations and multigenerational studies that are possible in microcosm experiments mean we can directly address some of the questions surrounding this new area of conservation which would be impossible at full size. 

Fundamental questions in community ecology can be fruitfully addressed through the frame of rewilding. 
Certainly, we’ll need to fill these gaps in our knowledge if we are to improve rewilding campaigns. 
We can propose three hypotheses which represent open ecological questions that have a significant bearing on rewilding. 

1.	Habitat size and complexity affects rewilding success
Connectivity and core area have been pointed out as being the most important considerations for rewilding campaigns \citep{soule1998rewilding}. 
Because of the large ranges of many species, a suitable habitat core is necessary to contain them. And some measure of connectivity between core sites allows the flow of animals between. 

2. The order in which species are reintroduced affects rewilding success 
Can we identify the best order with which species are reintroduced in order to ensure they successfully re-establish? 

3. The time between species extinction and species reintroduction affects rewilding success 
Time is an essential aspect to consider because rewilding proposals have huge variation with respect to the age of the system or the species that is being considered for reestablishment. 
Can a species re-establish itself when its habitat has evolved without it?

There has been a credibility gap in the scientific community with respect to microcosm work \citep{benton2007microcosm}. 
With critics arguing the results obtained at the small scale lose relevance when extrapolated up to larger systems \citep{carpenter1996microcosm}. 
However, this criticism has eroded in the face of impressive results derived from microcosm experiments with both an ecological and evolutionary relevance \citep{jessup2004big,benton2007microcosm,buckling2009beagle,mcclean2015single}.  


\section*{Acknowledgments}

A lot of people are to thank here.


\newpage


\bibliography{bibfile}


\newpage


\section*{Tables}


%\begin{table}[h!]
%  \caption{A first table caption.}
%  \label{Tab1}
%  \begin{center}
%    \begin{tabular}{p{3cm}p{10cm}}
%      Name & Description \\
%      \hline
%      Agri & Proportion of agricultural areas \\
%      Alpine & Proportion of alpine areas \\
%      Bare & Proportion of bare ground \\
%      DEM & Mean elevation \\
%      DEMslope & Mean slope \\
%      \hline
%    \end{tabular}
%  \end{center}
%\end{table}


\newpage


%\begin{table}[h!]
%  \caption{A second table caption, longer than the first one that was
%quite short. Indeed, it was supposed to be short, at the contrary of this one which is
%much more informative than the previous one.}
%  \label{Tab2}
%  \begin{center}
%    \begin{tabular}{lrrr}
%      Name & Mar & Spe1 & Spe2 \\
%      \hline
%      Agri & -0.050 & 0.026 & 0.173 \\
%      Alpine & -0.874 & -0.139 & 0.184 \\
%      Bare & -0.555 & -0.922 & 0.084 \\
%      DEM & -0.796 & -0.100 & 0.095 \\
%      DEMslope & -0.167 & -0.205 & 0.013 \\
%      \hline
%    \end{tabular}
%  \end{center}
%\end{table}


\newpage


\section*{Figures}


%\begin{figure}[h!]
%  \caption{What a nice figure\dots}
%  \label{Fig1}
%  \begin{center}
%    \includegraphics[width=6cm]{Fig1}
%  \end{center}
%\end{figure}


\newpage


%\begin{figure}[h!]
%  \caption{Even nicer!}
%  \label{Fig2}
%  \begin{center}
%    \includegraphics[width=6cm]{Fig2}
%  \end{center}
%\end{figure}


\end{document}
